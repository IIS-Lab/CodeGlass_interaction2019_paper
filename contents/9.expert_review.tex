Our user evaluation with novice developers (i.e., university students) confirms the benefits of CodeGlass.
To validate the potential of our system for professional use, we conducted informal expert reviews.
We recruited 8 expert programmers (PD1--PD8, all male) in 4 different IT companies.
Our interviewees agreed on benefits of CodeGlass the lab study participants also mentioned: support examining past pull requests (8 participants); facilitate developer onboarding (5 participants); and comprehend open source code (4 participants).
Beside these benefits, they expressed potential advantages for professional use.

\subsection*{Understanding Hidden Development Context}

In real development environment, programmers do not always achieve the most efficient coding due to various reasons (e.g., tight deadlines and lack of coding skills).
Raw code does not necessarily offer such backgrounds when developers revisited.
PD3 and PD7 stated that CodeGlass can help them obtain such past backgrounds of source code.

% \myquote{昔のリポジトリのメンテナンスとかしてると結構危ないコードがあって、でも当時のチーム構成や技術力だったりリリースしないといけないみたいなのを考えると仕方なかったりするんですよね。そういうのを過去のログ(プルリク)から読み取れていい。}{P3}

%\myquote{When I am taking care of old repositories, I often find pretty dangerous code. But it could not be helped because we didn't have enough team members and skills at that time. This [CodeGlass] can help me view such background stories from past logs, and I like it.}{PD3}

% \myquote{新規で入ってコード見ると、だめなコードがたくさんあるんですよね。でも実はそれは特定の技術を使ってはいけないみたいな制約のもとの苦渋の決断だったことってかなりあるんですよね。そういうのはコードをみても絶対に分からないし、かといって大規模だと履歴を追うのは相当つらい}{P7}

\myquote{Horribly-written code is often a product of tough decisions by various constraints. Such backgrounds would never be visible from code, but it is tedious to review histories in a large-scale project.}{PD3}

Although projects have specifications, they were changed over code revisions.
As a result, such specifications do not offer correct information about existing code.
Pull requests extracted by CodeGlass would be useful to fill the discrepancy between outdated specifications and code.

% \myquote{コードが仕様と合わなくなったり逆に仕様書とか読んだら間違ったりとかあるので、コードが一番大事というか、正確で、ただそのコードがなぜ入ったかっていうのは昔のプルリクエストで見るんで、そういう時に使えると思います。}{P3}

% \myquote{こういう大きい会社ではいろんな人がすごいスピードで開発しててドキュメントもないから、全体的なコード理解にはすごくいいと思う。}{P6}

% \myquote{仕様書をもとにコードを作って、仕様書をもとにテストするんですが、でもやっぱり経緯はわかんないし、仕様書とコードの間のギャップって凄まじいんですよね。仕様書もコードも結果であって、開発理由とかは本当にわからない}{P7}

\myquote{We write and test code based on a specification. But it doesn't tell me how the code has been developed. And a gap between the specification and actual code is huge. Both specifications and code are just end products, and they don't tell me reasons why they are here.}{PD7}


\subsection*{Documentation through Pull Requests}

Although the importance of software documentation is well known, developers do not spare time and effort for it in reality~\cite{A_Study_of_the_Documentation_Essential_to_Software_Maintenance}.
%The participants also stated that they do not create detailed documentation, and pull requests can be a useful information resource to understand the rationale of source code.
PD5 and PD8 shared with us a unique idea that CodeGlass could encourage them to write detailed pull requests for future revisitation.

% \myquote{これ見て思ったのは、後からプルリクを分かりやすいように書き直したいと思うだろうなと思って、情報をどんどんあとから追加してWikiとかドキュメントみたいにできるかも。}{P5}

% \myquote{ソースコードのドキュメントって最近廃れてきてる気がしていて、よく関数名の上にコメント書くとドキュメントができたり、変数名からドキュメント作るやつありますけど、結局面倒とかqualityが低いとか、あと情報量がないので使わないんですよね。何してるかなんてある程度コード見たらわかるのでどうでもよくて、やっぱりなぜそのコードになったのかが重要で、プルリクは開発フローに既にあるし、それをもう少しちゃんと書いたらドキュメントになるってのはすごく新しいし実用的ですね}{P8}


\myquote{It's important to share why we wrote this particular code. Because pull requests are already in our development process, and (with CodeGlass) if we write them a little more properly, they become documentations. That's pretty new and practical.}{P8}

% \subsection*{Assisting to Understand Codes in Open Source Projects}

% CodeGlass works on all repositories on GitHub by only cloning them to the server system.
% Developers often investigate open source projects to understand external libraries they use in their projects.
% They also see open source as a means of learning good coding practices.
% CodeGlass has a great potential for supporting code understanding in open source projects by using such resources.
% The participants agree on this as a potential use of CodeGlass.

% \myquote{僕がオープンソースを読むのは、一個はちょっと参考になるとか勉強しようとか、なんかメジャーなリポジトリがどう作られてるかとか勉強目的で読むのと、あとはうちが使ってるライブラリとかあるんですよね、で、それらが挙動がよくわかんない時とか、より深く見ていきたい時とかっていうのは、なんでこういう実装というか、意図はなんなんだみたいな、深く見る時に使うかなあと思います。}{P1}


% \myquote{オープンソースだったらめっちゃ喜ばれそうですねえ、僕らは大体使ってるライブラリに何か問題があった時に、見に行ったりするので、めっちゃ助かりますね。}{P4}


\subsection*{Supporting Code Review}

Code review is a common practice in professional development environment for quality control.
But proper code review requires deep comprehension of revisions and their reasons.
%Reviewers also have to acquire the context and history of the project to fully understand the changes.
3 participants mentioned that CodeGlass would be helpful in code review because it offers quick access to relevant past pull requests.

% \myquote{レビューしていてコードの意図がわかんない時とか、かなり複雑なコードだった時に、プルリクに詳しく説明があるかなあって感じだったので、今言ったような時に使うかなあと}{P1}

\myquote{When I don't understand the intention of the given code or when it is very complex, I expect there may be some details in pull requests, so it [CodeGlass] could be useful for that.}{PD1}

% \myquote{これを使えばプルリクの履歴でまあそれを探し出せる、もちろんレビューの時に情報は多いほうがいいので。}{P3}
%\myquote{With this [CodeGlass], I can find (information about context) from past pull requests. It's always good to have as much information as possible when I do code review.}{PD3}

% \myquote{レビューしててそもそもこのクラス全体は何用だっけとかなるので、レビュー中は特に大雑把にコードの意味を思い出したいことがたしかに結構ある。}{P6}


