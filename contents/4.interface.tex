We develop an interactive piece-level code examination tool, called CodeGlass.
A code piece is defined as any of consecutive lines in code.
It can be a single line, function, class or even the entire code in a file.
CodeGlass extracts the revision history on the selected code piece from commit logs.
Its interface then provides an overview that shows past pull requests associated with the selected code piece which contain descriptions, commits, and review comments.


CodeGlass also allows developers to examine past pull requests and commits on any line of code, not limited to meaningful chunks (e.g., classes or functions).
This capability is useful when developers need to investigate large classes and functions because they can have freedom to break down into smaller portions for their examination.

% GitHub displays a source code in a file in the red frame in Figure~\ref{fig:WebInterface}.
% CodeGlass provides ``Fetch'' button~(i.e., blue frame in Fig.~\ref{fig:WebInterface}) and a feature that the user can select codes by pointer~(i.e., green frame in Fig.~\ref{fig:WebInterface}).
Figure~\ref{fig:WebInterface} shows the CodeGlass interface.
The user specifies a code piece on which she desires to read descriptions and review comments of past pull requests simply by a mouse drag~(highlighted in light green in Figure~\ref{fig:WebInterface}).
CodeGlass displays a series of relevant pull requests in the chronological order as shown in Figure~\ref{fig:WebInterface}a.
Each pull request is summarized in a separate box ((1) in Figure~\ref{fig:WebInterface}a).
Each summary includes the title of a pull request, its ID, and the numbers of commits, lines and files changed, and comments.
It also contains links to the original pull request page, the diff summary, review comments, and the author's page with which users can further investigate details.
In addition, CodeGlass can also suggest other potentially-relevant commits if there exists any ((2) in Figure~\ref{fig:WebInterface}a).
When clicking one of them, the user can view the version associated with the selected commit and further investigate how code has been developed.
When the user clicks the title of a pull request summary, CodeGlass presents its review comments ((3) in Figure~\ref{fig:WebInterface}b).

      
The current CodeGlass system takes a server-client implementation.
The CodeGlass interface is implemented as a Google Chrome extension.
When the user selects a code piece, the interface sends the repository name, file name, and line numbers of the selected code piece to the server.
The server then estimates how the given code piece has been changed with the DiffTrack algorithm (explained later).
It extracts commits in which the selected code piece had revisions. 
The server then obtains pull requests containing these commits through GitHub API, and sends them back to the interface.

Our current prototype would function for any public repository on GitHub because the backend algorithm is language-agnostic as we explain later.
With proper permission settings, CodeGlass would be available even for private repositories.
% There are 57 million active repositories as of April 2017\footnote{\url{https://github.com/blog/2345-celebrating-nine-years-of-\\github-with-an-anniversary-sale}}.

