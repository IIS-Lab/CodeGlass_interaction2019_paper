\section{結論}

本研究では,コード断片に関連する情報を提供するCodeGlassというシステムを開発した.
CodeGlassはユーザが選択したコード断片に関連する過去のプルリクエストを抽出し,ユーザに提供する.
さらに,プルリクエストの説明文を解析し,実装内容や開発背景に関する文章をインターフェース上で強調して表示することができる.
情報系の学生およびプログラマを対象としたユーザ評価を行った結果,プルリクエストの説明文に含まれる情報が,コード断片を理解する上で有用であることが明らかとなった.

% We presented CodeGlass, an interface that provides pull requests and their review comments associated with a code piece.
% %We also developed a piece-level diff backtrack algorithm, called \shibato{}{DiffTrack}.
% Given a code piece, our system identifies its location in an older version even if changes were made.
% It precisely and accurately extracts a series of commits that include changes on the chosen code piece.
% CodeGlass then shows a series of pull requests containing these commits, and supports user's comprehension on development backgrounds.
% Our evaluations confirm the benefits of CodeGlass both quantitatively and qualitatively.
% Future work should examine such advantages in different settings of software development.

% \section*{Acknowledgements}
% (removed for review.)
% % We thank Arissa Sato for proofreading our paper.

\section*{謝辞}

本研究の一部は,科学研究費助成事業(若手研究18K18088)によって支援された.
また,本研究の実験に参加してくださった全ての実験参加者に感謝する.
% 最後に,本論文の執筆にあたり校正をしてくださった研究室のメンバーに心から感謝する.